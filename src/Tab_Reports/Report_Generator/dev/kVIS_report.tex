\documentclass[11pt,a4paper]{article}
\usepackage[utf8]{inputenc}
\usepackage{amsmath}
\usepackage{amsfonts}
\usepackage{amssymb}
\usepackage{graphicx}
\usepackage{color}
\usepackage[left=2.5cm,right=2.5cm,top=3cm,bottom=3cm]{geometry}
\usepackage{hyperref}

\author{Dr Kai Lehmkuehler}
\title{Lilium SysID Requirements and Flowchart\\ V1.2}
\begin{document}
\maketitle
\tableofcontents

\clearpage
\section{Introduction}
This document briefly lists the big picture layout of the Lilium sysID Framework, the workflow and a proposed number of steps for their implementation and the familiarisation with the Lilium Jet flight dynamics. Figure \ref{flowchart} shows the proposed workflow and the required components.

\begin{figure}[!ht]
\center
\includegraphics[width=\textwidth]{SysID_flow_chart.pdf}
\caption{SysID Process flow chart}
\label{flowchart}
\end{figure}

\clearpage
\section{Components and Workflow}

\subsection{Test Objectives}
To be determined in cooperation with the other departments

\subsection{Input Design}
To excite the required aircraft dynamics, design in cooperation with flight control and operations to ensure safety.

\subsection{Instrumentation}

\begin{itemize}
\item Minimum: Accel, Gyro, Attitude (Magnetometer, AHRS), GPS (min 10Hz, ideally 50 Hz), Airdata ($V_{air}$ ($\bar{q}$), $\alpha$, $\beta$, Temp, static P), Control Feedback (Motor RPM, Flap angles)
\item Sample rate: min. 50 Hz
\item Sample synchronity: better than 5ms, ideally 1ms
\item Time vector with true sample times (at least 0.1 ms resolution)
\item Must record all raw sensor data in addition to processed data 
\item Sensors must be aligned with aircraft body axes. Accelerometer position with respect to CG must be known. If large, corrections are required, which require measurements of angular acceleration. Airdata sensor location with respect to CG must be known.
\item Sensor calibrations must be available for data post processing
\item All CAN bus sensor delays must be characterised (Engine RPM)
\item As discussed, analog flap angle sensors with direct link to FTI are preferred
\item If initially only a single flap feedback per wing is possible, there should be an inboard sensor on one surface and an outboard sensor on the opposite surface
\end{itemize}


\subsection{Flight Test}

\begin{itemize}
\item Need: Reference parameters, CG position (all 3 axes), Mass properties, ideally weather (wind) information
\item Data acquisition package must be able to set markers in the data to identify manoeuvres (see Section \ref{fti_req})
\item Flight controller must be able to replay pre-recorded input sequences (see Section \ref{fti_req})
\item Data display on GCS: tbd.
\end{itemize}

\textbf{Are inertial properties affected by added mass?}

\textbf{Any merit to performing semi real-time sysID during flight (Morelli method)?}

\subsection{Data Download and \color{red}{File Structure}}

\begin{itemize}
\item Download the data for the aircraft and convert it into the post processing file structure.
\item Save all reference quantities and mass properties in this file, together with all applicable sensor calibrations and a flight log.
\end{itemize}

\subsection{Flight Path Reconstruction}

\begin{itemize}
\item Use an Extended Kalman Filter to estimate the true flight path, together with sensor errors, wind, and other disturbances.
\item Sensor calibrations shall be applied during the filter run for maximum fidelity.
\item Results to be saved into the data structure, while preserving the original flight data, for later comparison.
\item Further, assemble \texttt{fdata} structure for sys ID toolbox.
\item Save everything into the flight data base.
\end{itemize}

\subsection{\color{red}{Flight Data Base}}
How to store the data in a useful manner? Such that things can be found, searched and cross-referenced? Store full flight only or also store snippets for easy access during later processing?

\subsection{Data Edit}

\begin{itemize}
\item Extract required manoeuvre data from one or more flight data set(s) for analysis
\item Preserve full data structure to be able to inspect all aspects of the flight condition 
\end{itemize}

\subsection{SysID}

\begin{itemize}
\item Use model structures and algorithms to extract model parameters from flight data
\item Requires batch run capability
\item Must be able to run data from several flights in a single batch
\end{itemize}

Methods available: EQN, OEM, FEM

\textbf{Potentially: OEKID (in development)}

\subsection{SysID Results}

\begin{itemize}
\item Plot and print results together with quality indicators and error estimates
\item Validate results by comparing simulation results against flight data
\item Save results into the aerodynamic database
\item Use results to define objectives of the next flight test
\end{itemize}

\clearpage
\section{Development / Familiarisation Procedure}

\subsection{Step 1}

\begin{itemize}
\item Define sensor suite
\item Define file format
\item Create or adapt a flight simulation that outputs the sensor readings in the defined file format, plus ability to specify sensor errors, wind, turbulence and airframe vibrations, as well as changes to aerodynamic- and mass properties.
\end{itemize}

\subsection{Step 2}

\begin{itemize}
\item Adapt data management GUI to the defined file format
\item Adapt flight path reconstruction to new file format and sensor configuration
\item Adapt sysID scripts to new file format
\end{itemize}

\subsection{Step 3}

\begin{itemize}
\item Define initial objectives of sysID
\item Define suitable model structures and input sequences
\item Verify with simulation data, including sensitivities to noise and errors
	\begin{itemize}
	\item Which methods are suitable?
	\item What model parameters are important, and which not?
	\item Any obvious correlation issues?
	\item How do the results deteriorate with noise, turbulence and other errors?
	\end{itemize}
\end{itemize}

\subsection{Step 4}

\begin{itemize}
\item Develop the flight data base (design and implementation)
\end{itemize}


\clearpage
\section{Flight Test System Requirements}
\label{fti_req}

\subsection{Data Recording}

Marker in the data corresponding to the flight test plan enable easy searching and indexing of the flight test data and are a standard feature in industry. These markers should indicate the start and stop time of the manoeuvre, as well as an unique identifier.

\subsection{Input Generation}

Automated control input generation is required for several aspects of the data gathering for system ID purposes:

\begin{itemize}
\item Remote pilots have difficulty in manually generating suitable inputs due to the limited feedback via the telemetry link only.
\item Complex input shapes and/or multi-axis synchronised inputs can improve signal content and flight safety, especially with a closed loop flight control system. Examples are orthogonal multi-sines, which effectively excite the aircraft dynamics while keeping the overall departure from the trim condition within limits. Such inputs are also required identify control derivatives in scenarios, where multiple controls have similar effects.
\item Inputs are repeatable, and can be modified in a controlled manner.
\item Inputs can be evaluated prior to the flight in simulations and then are guaranteed to be executed correctly during flight
\end{itemize}

The following input generation system has been successfully used for my flight test work, and aligns well with other company's approaches:

\begin{figure}[!ht]
\center
\includegraphics[width=\textwidth]{input_gen.pdf}
\caption{Input generator logic}
\label{input_gen}
\end{figure}

\newpage
Inputs are generated as commanded from the ground station from a pre-defined waveform, that has been loaded into the aircraft prior to flight. The commands include the input number, as well as speed, direction and magnitude modifiers to the input shape. This allows for quick modification of the pre-loaded input to react to the dynamic responses during the flight, i.e. increase the magnitude if the baseline input definition was too conservative. The generated input waveform runs through a limiter to protect against errors of the generator. The signal then passes through the protection logic, which checks the aircraft flight condition, and terminates the input sequence if pre-defined limits in attitude, airspeed or altitude are violated. The input commands are then added as a disturbance to the actuator command signal.

As an example, a sample input definition file used by my system is shown below:

\begin{verbatim}
# UAVmainframe manoeuvre input definition file

# Hold time at the end for motion dissipation [ms]
hold=1000

# Specify input type [valid: doublet, m_doublet, sine] - used to enable sequence 
# manipulation functionality
# doublet: single axis square wave, m_doublet: multi axis square wave, 
# sine: long, continuous waveform
Sequence=doublet

# Input steps (resolution 10ms) = +- % of trim control deflection  
# Max 1024 steps at this time - exceed at own risk...
# Parameters: time [ms] = Elev, Ail, Rudd, Throttle, Aux1, Aux2
0   = 0,0,0,0,0,0
1000= 7,0,0,0,0,0
1750=-10,0,0,0,0,0
2250= 12,0,0,0,0,0
2500=-15,0,0,0,0,0
2750= 0,0,0,0,0,0
4750=-7,0,0,0,0,0
5500= 12,0,0,0,0,0
6000=-10,0,0,0,0,0
6250= 10,0,0,0,0,0
6500= 0,0,0,0,0,0
8000= 0,0,0,0,0,0

# need to have one zero step at the end for inverting algorithm
#end input file
\end{verbatim}

\end{document}